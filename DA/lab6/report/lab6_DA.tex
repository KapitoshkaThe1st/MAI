 
 
 
\documentclass[12pt]{article}

\usepackage{fullpage}
\usepackage{multicol,multirow}
\usepackage{tabularx}
\usepackage{ulem}
\usepackage[utf8]{inputenc}
\usepackage[russian]{babel}
\usepackage{graphicx}
\usepackage{float}

\begin{document}

\section*{Лабораторная работа №\,6 по курсу дискрeтного анализа: Калькулятор. Длинная арифметика}

Выполнил студент группы 08-208 МАИ \textit{Куликов Алексей}.

\subsection*{Условие}

Необходимо разработать программную библиотеку на языке C или
C++, реализующую простейшие арифметические действия и проверку
условий над целыми неотрицательными числами. На основании этой
библиотеки, нужно составить программу, выполняющую вычисления
над парами десятичных чисел и выводящую результат на стандартный
файл вывода.
Список арифметических операций:
\begin{itemize}
    \item Сложение (+).
    \item Вычитание (-).
    \item Умножение (*).
    \item Возведение в степень (\verb|^|).
    \item Деление (/).
\end{itemize}

В случае возникновения переполнения в результате вычислений,
попытки вычесть из меньшего числа большее, деления на ноль или
возведении нуля в нулевую степень, программа должна вывести на
экран строку Error.
Список условий:
\begin{itemize}
    \item Больше (>).
    \item Меньше (<).
    \item Равно (=).
\end{itemize}
В случае выполнения условия, программа должна вывести на экран
строку true, в противном случае — false.

\subsection*{Метод решения}

Простейшие операции сложения и вычитания полность повторяют процедуру сложения или вычитания <<в столбик>> на листе бумаги. Т.е. начиная с младших разрядов складываем или вычитаем разряды, где надо запоминаем избыток (для сложения), или <<занимаем>> (для деления) и учитываем это во время операции над следующим разрядом. Для сложения, если после прохода по всем разрядом остался избыток, то добавляем еще один разряд и избыток записываем туда.

Чуть сложнее операция умножения, и, кроме того, она слегка отличается от процедуры умножения <<на листочке>>. Сдесь суммиррование по результатам перемножений разрядов происходит сразу, и, если возникает избыток, учитываем его на следующем этапе. Так же, если после прохода по всем разрядам, остался избыток, то приписываем его в конец числа-результата.

Деление осуществляется по-сути так же, как <<в столбик>>, только здесь необходим алгоритм для угадывания числа, на которое надо домножить делитель, чтобы оно вместилось в оставшееся от делителя к текущему этапу. Для этого здесь использован бинарный поиск. В остальном процедура деления та же.

Возведение в степень реализовано бинарно.

Операции сравнения предельно просты. Просто идем от младших разрядов к старшим, до первого несовпадения и в зависимости от отношения этих двух результатов возвращаем результат. Так же не имеет смысла сравнивать числа разной длины, поэтому уже из этого можно делать вывод об отношении чисел.

\subsection*{Описание программы}

Программа состоит из единственного исходного файла. В нем определен класс \verb|TBigInt|, реализующий арифметические операции над длинными числами.

Имеется парочка конструкторов (из строки или 16-разрядного инта), опрератор вывода, всевозможные арифметические операции: сложение, вычитание, умножение, целочисленное деление, возведение в степень (длинное-длинное) и деление длинного на короткое, и возведение в короткую степень (это сделано скорее для ускорения работы длинных операторов). Все операторы реализованы вышеописанным образом.

\subsection*{Дневник отладки}

Особых проблем не возникало, программа прошла проверку с первого раза.

\subsection*{Тест производительности}

\begin{table}[H]
\caption{Скорость выполнения арифметических операций}
\label{tabular:timesandtenses}
\begin{center}
\begin{tabular}{|l|c|}
\hline
\textbf{Тип операции} & \textbf{Время работы (мкс.)} \\
\hline
Сложение & 3 \\
\hline
Вычитание & 3 \\
\hline
Умножение & 12 \\
\hline
Деление & 90 \\
\hline
Возведение в степень & 44379 \\
\hline
\end{tabular}
\end{center}
\end{table}

Сложение, вычитание, умножение и деление измерялись для чисел порядка $10^{94}$.
Возведение степень измерялись с основанием порядка $10^5$ и показателем степение порядка $10^4$.

\subsection*{Выводы}

Длинная арифметика может применяться в криптографии. Большинство систем подписывания и шифрования данных используют целочисленную арифметику по модулю m, где m — очень большое натуральное число, не обязательно простое. Например, при реализации метода шифрования RSA, криптосистемы Рабина или схемы Эль-Гамаля требуется обеспечить точность результатов умножения и возведения в степень порядка $10^{309}$ (сам пока не сталкивался, но верю Википедии на слово).

В целом, большой сложности реализация базовой длинной арифметики над целыми числами без каких-то хитрых оптимизаций сложности не представляет. Более быстрые же реализации требуют более сложных алгоритмов, которых, вероятно, существует даже несколько. К счастью их реализовывать пока не пришлось.
\end{document}
