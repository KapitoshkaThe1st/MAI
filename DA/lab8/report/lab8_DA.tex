\documentclass[12pt]{article}

\usepackage{fullpage}
\usepackage{multicol,multirow}
\usepackage{tabularx}
\usepackage{ulem}
\usepackage[utf8]{inputenc}
\usepackage[russian]{babel}
\usepackage{graphicx}
\usepackage{float}

\begin{document}

\section*{Лабораторная работа №\,8 по курсу дискрeтного анализа: Жадные алгоритмы}

Выполнил студент группы 08-208 МАИ \textit{Куликов Алексей}.

\subsection*{Условие}

Разрабтать жадный алгоритм решения задачи, определяемой своим
вариантом. Доказать его корректность, оценить скорость и объём
затрачиваемой оперативной памяти.
Реализовать программу на языке C или C++, соответствующую
построенныму алгоритму. Формат входных и выходных данных описан в варианте задания.

В качестве конкретного задания предлагается решить следующую задачу (вариант 6): 

Заданы N объектов с ограничиениями на расположение вида «A должен
находиться перед B». Необходимо найти такой порядок расположения
объектов, что все ограничения будут выполняться. Входные данные: на
первой строке два числа, N и M, за которыми следует M строк с
ограничениями вида «A B» ($1 \leq A$, $B \leq N$ ) определяющими
относительную последовательность объектов с номерами A и B.
Выходные данные: -1 если расположить объекты в соответствии с
требованиями невозможно, последовательность номеров объектов в
противном случае.

\subsection*{Метод решения}

Задачу можно представить в виде набора процессов. Каждый процесс может требовать (а может и нет) выполнения некоторого набора процессов до него. 

Эта зависимость представляется в виде графа, в вершинах которых номера процессов, а ребра обозначают зависимости между процессами.

Жадная идея здесь состоит в том, что на каждом шагу выбирается вершина, не имеющая входящих ребер. Т.о. такая вершина либо "не требует" выполнения никаких других процессов, либо они уже выполнены (вершина удалена из графа вместе с исходящими ребрами). Поэтому ее можно беспрепятственно взять, положить следующей в порядок выполнения (все предшествующие вершины уже лежат там). Далее все поворяется, пока возможно выбрать вершину, не имеющую входящих ребер. Если такую вершину нельзя выбрать из-за того, что все вершины обработаны и удалены, то алгоритм отработал успешно и требуемая последовательность вершин составлена. Если же остались необработанные вершины и, при этом, нет ни одной, не имеющей входящих ребер, то в графе зависимостей присутствует цикл, и его нельзя топологически отсортировать. 

\subsection*{Описание программы}

Программа состоит из единственного исходного файла. В нем записано рекурсивное решение согласно алгоритму, описанному выше, с применением мемоизации.

\subsection*{Дневник отладки}

Особых проблем не возникало, программа прошла проверку с первого раза.

\subsection*{Выводы}

Жадные алгоритмы -- это алгоритмы, которые на каждом этапе выбирается локально оптимальное решение, надеясь на то, что и решение всей задачи окажется оптимальным. Многие задачи могут быть успешно решены с помощью жадных алгоритмов, причем быстрее, чем другими методами.

Конкретно данный алгоритм может иметь довольно много практических применений. Его можно использовать везде, где нужно построить последовательность выполнения действий, каждое из которых может зависеть от других. Подобное требуется, например, при установке программ при помощи пакетного менеджера, сборки исходных текстов программ при помощи Makefile'ов и т.п. 

\end{document}
