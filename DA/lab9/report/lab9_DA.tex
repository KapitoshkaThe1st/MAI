\documentclass[12pt]{article}

\usepackage{fullpage}
\usepackage{multicol,multirow}
\usepackage{tabularx}
\usepackage{ulem}
\usepackage[utf8]{inputenc}
\usepackage[russian]{babel}
\usepackage{graphicx}
\usepackage{float}

\begin{document}

\section*{Лабораторная работа №\,9 по курсу дискрeтного анализа: Графы}

Выполнил студент группы 08-208 МАИ \textit{Куликов Алексей}.

\subsection*{Условие}
Разработать программу на языке C или C++, реализующую указанный
алгоритм. Формат входных и выходных данных описан в варианте
задания. Первый тест в проверяющей системе совпадает с примером.

 В качестве конкретного задания предлагается решить следующую задачу (вариант 4): 

Задан взвешенный неориентированный граф, состоящий из $n$ вершин и $m$
ребер. Вершины пронумерованы целыми числами от $1$ до $n$. Необходимо
найти длину кратчайшего пути из вершины с номером \verb|start| в вершину с
номером \verb|finish| при помощи алгоритма Дейкстры. Длина пути равна
сумме весов ребер на этом пути. Граф не содержит петель и кратных
ребер.

\subsection*{Метод решения}

Решением является стандартный алгоритм Дейкстры с использованием очереди с приоритетами.

На каждом этапе выбирается такая вершина \verb|minDistVert|, которая не использовалась ранее и длина пути \verb|dist[minDistVert]| от которой до стартовой вершины наименьший (такую вершину можно быстро получить из очереди с приоритетом). Далее для каждой из смежных с ней вершин \verb|to| запускаем процедуру релаксации. Релаксация -- по сути, улучшение текущего решения, если это возможно. Процедура релаксации: \verb|dist[to] = min(dist[to], dist[minDistVert] + len)|, где \verb|len| -- длина дуги от вершины \verb|minDistVert| до рассматириваемой смежной \verb|to|.

Релаксации повторяются до тех пор, пока не будут пройдены все вершины, либо пока не останется вершин, удаленных от стартовой менее чем на бесконечность. Это будет означать, что такие вершины по-просту никак не связваны со стартовой и из нее до подобных вершин не добраться.

В результате в \verb|dist[i]| будет храниться длина кратчайшего пути от вершины \verb|start| до \verb|i|-й вершины. 

\subsection*{Описание программы}

Программа состоит из единственного исходного файла. В нем записано решение согласно алгоритму, описанному выше.

\subsection*{Дневник отладки}
\begin{enumerate}
    \item 07.05 Не проходит тест номер 14. \\
    РЕШЕНИЕ: Пришел к выводу, что после удачной релаксации одного из путей ломалась куча из-за того, что функция-компаратор использовала актуальные данные о кратчайших расстояниях, которые в свою очередь успевали меняться до того, как до соответствующей вершины могла бы дойти очередь. Переписал с использованием данных на момент добавленийя. Т.о. вновь добавляемое состояние для вершины может иметь только меньшее значение расстояния, чем предыдущее, и значит, окажется выше в куче, будет взята раньше, а каждую вершину обрабатываем только раз (помечаем в векторе \verb|used|). Т.о. алгоритм всегда выбирает вершину с наименьшим текущим расстоянием от старта.
    \item 07.05 Не проходит тест номер 7. \\
    РЕШЕНИЕ: Ошибся с компаратором и строил кучу не по убыванию, а по возрастанию. Заменил компаратор на \verb|std::greater|.

\end{enumerate}

\subsection*{Выводы}

В решении данной задачи самым трудным оказалось понять ошибку с функцией-компаратором, использующей актуальный вектор дистанций. Оказвается свежее не всегда лучшее.

Алгоритм Дейкстры может быть использован везде, где нужно найти кратчайший путь от вершины до вершины (вершин). Например, в для нахождения кратчайших маршрутов между точками на карте. Так же может быть использован в системах коммуникации для IP маршрутизаци и т.д. и т.п.

\end{document}
